\documentclass[a4paper,11pt]{article}
\usepackage{amsmath,amssymb,amsthm, tikz,titlesec,hyperref,esint,braket, graphicx}
\usepackage[a4paper,margin=2cm]{geometry}
\linespread{1.3}
\newtheorem{claim}{Claim}[section]

\newcommand\name{Park Chanwoo}   % Name of the student
\newcommand\university{KAIST} % Name of the university
\newcommand\department{Physics} % Name of the department
\newcommand\studentid{20230297} % Student ID
\newcommand\s{\,\;}


\newenvironment{solution}[1]
  {\renewcommand\qedsymbol{$\square$}\begin{proof}[\textbf{Solution#1}]}
  {\end{proof}}
\newenvironment{note}
  {\renewcommand\qedsymbol{$\blacksquare$}\begin{proof}[\textnormal{\textbf{note}}]}
  {\end{proof}}

\title{KAIST\\2025 PH475 Quantum Information I\\
Homework 1\bigskip}
\author{\textbf{\Large \name} \\
% University: \university\\
Department: \department\\
Student ID: \studentid}
\date{\today}

\begin{document}
\thispagestyle{empty}
\maketitle
\tableofcontents
\titleformat{\section}[frame]{\pagebreak}{\filright
\footnotesize  \enspace \textsf{KAIST --- PH475 Quantum Information I 2025 Spring}\enspace}{6pt}{\Large\bfseries\filcenter}

\newcommand{\der}[2][]{\frac{d #1}{d #2}}
\newcommand{\pder}[2][]{\frac{\partial #1}{\partial #2}}
\newcommand{\grad}{\operatorname{grad}}
\newcommand{\diver}{\operatorname{div}}
\newcommand{\curl}{\operatorname{curl}}
\newcommand{\tr}{\operatorname{tr}}

\newcommand{\cnot}{C_{\text{NOT}}}
\newcommand{\cz}{C_{Z}}

\section{\#1}


Let 
\begin{equation}
    \rho=\frac{1}{2}\left(I+\vec r\cdot\vec\sigma\right)=\frac{1}{2}\left(I + r_1\sigma_1+r_2\sigma_2+r_3\sigma_3\right)
\end{equation}

Note that(from the first homework of this semester)
\begin{equation}
    \sigma_i\sigma_j=\varepsilon_{ijk} i\sigma_k + \delta_{ij}I
\end{equation}

Since Pauli matrices are traceless, 
\begin{equation}
    \tr(\sigma_i\sigma_j) = \tr(\varepsilon_{ijk} i\sigma_k + \delta_{ij}I) = \varepsilon_{ijk} i\tr \sigma_k + \delta_{ij}\tr I = 2\delta_{ij}
\end{equation}

Hence for the Pauli operators $X=\sigma_1, Y=\sigma_2, Z=\sigma_3$,
\begin{align}
    \tr(\sigma_i\rho) 
    &= \tr\left[\frac{1}{2}\left(\sigma_i+\sum_jr_j\sigma_i\sigma_j\right)\right]\\
    &=\frac{1}{2}\left(\tr\sigma_i+\sum_jr_j\tr(\sigma_i\sigma_j)\right)\\
    &=\frac{1}{2}\left(0+\sum_jr_j(2\delta_{ij})\right)\\
    &=\sum_{j}r_j\delta_{ij}\\
    &=r_i
\end{align}

Therefore,
\begin{equation}
    \vec{r}=\left(\tr(\rho X), \tr(\rho Y), \tr(\rho Z)\right)
\end{equation}

The condition $\tr\rho=1$ is already satisfied.
\begin{align}
    \tr\left(\frac{1}{2}\left(I + r_1\sigma_1+r_2\sigma_2+r_3\sigma_3\right)\right)
    &=\frac{1}{2}\tr\left(I + r_1\sigma_1+r_2\sigma_2+r_3\sigma_3\right) \\
    &=\frac{1}{2}\left(2 + r_1\cdot 0+r_2\cdot 0+r_3\cdot 0\right) \\
    &=1
\end{align}

Let $\lambda_1$ and $\lambda_2$ be the eigenvalue of $\rho$ then,
\begin{align}
    \lambda_1 + \lambda_2 &= \tr\rho = 1\\
    \lambda_1\lambda_2&=\det\rho\\
    &=\frac{1}{4}\left((1+r_1)(1-r_1)-(r_2+ir_3)(r_2-ir_3)\right)\\
    &=\frac{1}{4}(1-r_1^2-r_2^2-r_3^2)\\
    &=\frac{1}{4}(1-|\vec r|^2)\\
    &=\left(\frac{1-|\vec{r}|}{2}\right)\left(\frac{1+|\vec{r}|}{2}\right)
\end{align}

Hence the eigenvalues are
\begin{equation}
    \lambda=\frac{1}{2}(1\pm|\vec{r}|)
\end{equation}

For $\rho$ to be positive semi-definite, the eigenvalues must be non-negative. Hence,
\begin{equation}
   \frac{1}{2}(1\pm|\vec{r}|)\geq 0\Rightarrow |\vec{r}| \leq 1 
\end{equation}

The purity of the density matrix is defined by
\begin{equation}
    \tr\rho^2=\lambda_1^2+\lambda_2^2=\frac{1}{4}(1+|\vec r|)^2+\frac{1}{4}(1-|\vec r|)^2=\frac{1}{2}(1+|\vec r|^2)
\end{equation}




\end{document}
