\documentclass[a4paper,11pt]{article}
\usepackage{amsmath,amssymb,amsthm, tikz,titlesec,hyperref,esint,braket, graphicx}
\usepackage[a4paper,margin=2cm]{geometry}
\linespread{1.3}
\newtheorem{claim}{Claim}[section]

\newcommand\name{Park Chanwoo}   % Name of the student
\newcommand\university{KAIST} % Name of the university
\newcommand\department{Physics} % Name of the department
\newcommand\studentid{20230297} % Student ID
\newcommand\s{\,\;}


\newenvironment{solution}[1]
  {\renewcommand\qedsymbol{$\square$}\begin{proof}[\textbf{Solution#1}]}
  {\end{proof}}
\newenvironment{note}
  {\renewcommand\qedsymbol{$\blacksquare$}\begin{proof}[\textnormal{\textbf{note}}]}
  {\end{proof}}

\title{KAIST\\2025 PH475 Quantum Information I\\
Homework 1\bigskip}
\author{\textbf{\Large \name} \\
% University: \university\\
Department: \department\\
Student ID: \studentid}
\date{\today}

\begin{document}
\thispagestyle{empty}
\maketitle
\tableofcontents
\titleformat{\section}[frame]{\pagebreak}{\filright
\footnotesize  \enspace \textsf{KAIST --- PH475 Quantum Information I 2025 Spring}\enspace}{6pt}{\Large\bfseries\filcenter}

\newcommand{\der}[2][]{\frac{d #1}{d #2}}
\newcommand{\pder}[2][]{\frac{\partial #1}{\partial #2}}
\newcommand{\grad}{\operatorname{grad}}
\newcommand{\diver}{\operatorname{div}}
\newcommand{\curl}{\operatorname{curl}}
\newcommand{\tr}{\operatorname{tr}}

\newcommand{\cnot}{C_{\text{NOT}}}
\newcommand{\cz}{C_{Z}}

\section{\#1}

\begin{note}
Definitions of controlled unitary operator
\begin{gather}
    \cnot\ket{0}\ket{0} = \ket{0}\ket{0},\quad
    \cnot\ket{0}\ket{1} = \ket{0}\ket{1},\quad
    \cnot\ket{1}\ket{0} = \ket{1}\ket{1},\quad
    \cnot\ket{1}\ket{1} = \ket{1}\ket{0}\\
    \cz\ket{0}\ket{0} = \ket{0}\ket{0},\quad
    \cz\ket{0}\ket{1} = \ket{0}\ket{1},\quad
    \cz\ket{1}\ket{0} = \ket{1}\ket{0},\quad
    \cz\ket{1}\ket{1} = -\ket{1}\ket{1}
\end{gather}

\end{note}

(a)
\begin{align}
    \cnot \ket{+}\ket{0}
    &=\cnot\left(\frac{1}{\sqrt{2}}\ket{0}\ket{0} + \frac{1}{\sqrt{2}}\ket{1}\ket{0}\right) \\
    &=\frac{1}{\sqrt{2}}\cnot\ket{0}\ket{0} + \frac{1}{\sqrt{2}}\cnot\ket{1}\ket{0} \\
    &=\frac{1}{\sqrt{2}}\ket{0}\ket{0} + \frac{1}{\sqrt{2}}\ket{1}\ket{1}
\end{align}

(b)
\begin{align}
    \cnot \ket{+}\ket{+}
    &=\cnot\left(\frac{1}{2}\ket{0}\ket{0} + \frac{1}{2}\ket{0}\ket{1} + \frac{1}{2}\ket{1}\ket{0} + \frac{1}{2}\ket{1}\ket{1}\right) \\
    &=\frac{1}{2}\cnot\ket{0}\ket{0} + \frac{1}{2}\cnot\ket{0}\ket{1} + \frac{1}{2}\cnot\ket{1}\ket{0} + \frac{1}{2}\cnot\ket{1}\ket{1} \\
    &=\frac{1}{2}\ket{0}\ket{0} + \frac{1}{2}\ket{0}\ket{1} + \frac{1}{2}\ket{1}\ket{1} + \frac{1}{2}\ket{1}\ket{0}\\
    &=\ket{+}\ket{+}
\end{align}

It is separable, not entangled.

(c)
\begin{align}
    \cz \ket{+}\ket{+}
    &=\cz\left(\frac{1}{2}\ket{0}\ket{0} + \frac{1}{2}\ket{0}\ket{1} + \frac{1}{2}\ket{1}\ket{0} + \frac{1}{2}\ket{1}\ket{1}\right) \\
    &=\frac{1}{2}\cz\ket{0}\ket{0} + \frac{1}{2}\cz\ket{0}\ket{1} + \frac{1}{2}\cz\ket{1}\ket{0} + \frac{1}{2}\cz\ket{1}\ket{1} \\
    &=\frac{1}{2}\ket{0}\ket{0} + \frac{1}{2}\ket{0}\ket{1} + \frac{1}{2}\ket{1}\ket{0} - \frac{1}{2}\ket{1}\ket{1}\\
    &=\frac{1}{\sqrt{2}}\ket{0}\left(\frac{1}{\sqrt{2}}\ket{0} + \frac{1}{\sqrt{2}}\ket{1}\right) + \frac{1}{\sqrt{2}}\ket{1}\left(\frac{1}{\sqrt{2}}\ket{0} - \frac{1}{\sqrt{2}}\ket{1}\right)\\
    &=\frac{1}{\sqrt{2}}\ket{0}\ket{+} + \frac{1}{\sqrt{2}}\ket{1}\ket{-}
\end{align}

\end{document}
